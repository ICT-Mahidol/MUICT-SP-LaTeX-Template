\chapter{Introduction}
\label{Ch:Introduction}
The first paragraph of each chapter in a research or academic document serves as an introduction to the content of that specific chapter. It provides a brief overview of what readers can expect to find in the chapter and helps establish the context for the topics that will be discussed. For example, Chapter \ref{Ch:Introduction} consists of motivations, objectives... etc.

\section{Problem Statement}
\label{Sec:ProblemStatement}

Writing a problem statement is a crucial step in research, as it defines the specific issue or challenge you intend to address with your study. A well-written problem statement sets the foundation for the rest of your research, guiding your focus, research design, and methodology. 

\begin{itemize}
  \item  Here is my Problem 1
  \item  Here is my Problem 2
  \item  Here is my Problem 3
  \cite{Spi2020}
\end{itemize}

\section{Objectives of the project}
\label{Sec:Objective}
The objectives of a research study outline the specific goals, outcomes, or intentions that the researcher aims to achieve through the research process. Objectives provide a clear roadmap for the study and help guide the research design, methodology, data collection, and analysis. They answer the question "What do you intend to accomplish with your research?" Here's how to formulate research objectives effectively:

\begin{itemize}
    \item Be Specific and Clear:
    Each objective should be specific and clearly defined. Avoid vague or broad statements that can be interpreted in multiple ways.

    \item Start with Action Verbs:
    Begin each objective with action verbs that describe what you plan to do. Common verbs include "investigate," "analyze," "compare," "examine," "identify," "evaluate," and "develop."

    \item Align with Problem Statement:
    Ensure that your objectives directly address the research problem or question you've identified in your problem statement.

    \item Be Realistic:
    Set objectives that are achievable within the scope of your research study. Avoid setting unrealistic or overly ambitious goals. 
\end{itemize}

\section{Scope of the project}
\label{Sec:ScopeOfWork}
The scope of research refers to the boundaries and limitations set for a particular study. It defines the extent of the study's coverage, including the specific aspects, variables, and parameters that will be included or excluded from the research. Clearly defining the scope helps researchers focus their efforts, manage resources, and ensure that the study remains manageable and achievable. 
\begin{itemize}
    \item Scope to do something
    \item Scope to do something
    \item Scope to do something
\end{itemize}

\section{Expected benefits}
\label{Sec:ExpectedBenefit}
The expected benefits of a project refer to the positive outcomes, improvements, or advantages that are anticipated as a result of successfully completing the project's goals and objectives. These benefits contribute to the project's overall value and justification. When outlining the expected benefits of a project, it's important to consider both tangible and intangible gains. 
\begin{itemize}
    \item Benefit something 
    \item Benefit something
    \item Benefit something
    \item Benefit something
\end{itemize}

